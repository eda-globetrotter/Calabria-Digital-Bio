%	This LaTeX file is written by Zhiyang Ong to record notes for his digital biology class regarding the basics of software engineering and the UNIX environment.

%	The MIT License (MIT)

%	Copyright (c) <2014> <Zhiyang Ong>

%	Permission is hereby granted, free of charge, to any person obtaining a copy of this software and associated documentation files (the "Software"), to deal in the Software without restriction, including without limitation the rights to use, copy, modify, merge, publish, distribute, sublicense, and/or sell copies of the Software, and to permit persons to whom the Software is furnished to do so, subject to the following conditions:

%	The above copyright notice and this permission notice shall be included in all copies or substantial portions of the Software.

%	THE SOFTWARE IS PROVIDED "AS IS", WITHOUT WARRANTY OF ANY KIND, EXPRESS OR IMPLIED, INCLUDING BUT NOT LIMITED TO THE WARRANTIES OF MERCHANTABILITY, FITNESS FOR A PARTICULAR PURPOSE AND NONINFRINGEMENT. IN NO EVENT SHALL THE AUTHORS OR COPYRIGHT HOLDERS BE LIABLE FOR ANY CLAIM, DAMAGES OR OTHER LIABILITY, WHETHER IN AN ACTION OF CONTRACT, TORT OR OTHERWISE, ARISING FROM, OUT OF OR IN CONNECTION WITH THE SOFTWARE OR THE USE OR OTHER DEALINGS IN THE SOFTWARE.

%	Email address: echo "cukj -wb- 23wU4X5M589 TROJANS cqkH wiuz2y 0f Mw Stanford" | awk '{ sub("23wU4X5M589","F.d_c_b. ") sub("Stanford","d0mA1n"); print $5, $2, $8; for (i=1; i<=1; i++) print "6\b"; print $9, $7, $6 }' | sed y/kqcbuHwM62z/gnotrzadqmC/ | tr 'q' ' ' | tr -d [:cntrl:] | tr -d 'ir' | tr y "\n"

%%%%%%%%%%%%%%%%%%%%%%%%%%%%%%%%%%%%%%%%%%%%%%





%%%%%%%%%%%%%%%%%%%%%%%%%%%%%%%%%%%%%%%%%%%%%%
\chapter{Software Engineering Basics}
\label{chp:SWEngrBasics}


%%%%%%%%%%%%%%%%%%%%%%%%%%%%%%%%%%%%%%%%%%%%%%
\section{UNIX Basics}
\label{sec:UNIXBasics}

My TA, Ricardo, suggested using Guake (\url{http://en.wikipedia.org/wiki/Guake}) as a substitute for the common/normal {\tt Terminal} application. \\

We will be using the {\tt Terminal} to do a lot of our work in this class. Prof. Rodolfo Aramayo briefly talked about the history of UNIX and its derivatives, such as {\it Linux}, {\it BSD}, {\it Oracle/SUN Solaris}, and {\it Mac OS X}. UNIX was started at {\it Bell Labs}. He also talked about the UNIX philosophy. \\

We shall operate in the UNIX environment via text files. Everything (including directories) is a file in UNIX. Some files can be read visually (i.e., text files), while others (i.e., binary files) cannot. \\




The kernel is the heart of the operating system. The UNIX shell (accessed via applications, such as the {\tt Terminal}) is an application that allows users to interact with the kernel indirectly. \\





Anatomy of UNIX commands: {\tt command\_name [options] [arguments]}. Double dashes for options of UNIX commands cannot be combined. However, for options for single dash lines, they can be combined. \\

The ``{\tt man}'' page is the UNIX manual. To find documentation of a UNIX command, use the {\tt man} command. \\




UNIX commands to learn: \vspace{-0.3cm}
\begin{enumerate}	\itemsep -4pt
\item alias: ``alias ll''
\item apropos: ``apropos copy'' would search the UNIX ``man'' pages for the keyword ``copy''.
\item cat: conCATenate
\item cd
\item chmod: Change mode
\item clear
\item cp: cp --version
\item dir -l
\item date
\item du: ``du -hd 0 .'' list the size of the directory in KB, and ``du -hd 1 .'' list the size of the directory and its files. ``df -h'' indicates the size of the directory and its contents. 
\item echo: ``echo -e'' refers to {\tt echo} enhanced, which redirects the output in the UNIX pipeline to a file. {\it echo -e `` `date`'' $>$ tata1}. ``echo \$PATH''
\item file
\item history
\item info cp
\item less
\item ls [-al]
\item more
\item mkdir
\item mv
\item pwd
\item rm
\item rmdir
\item rsync: \vspace{-0.3cm}
	\begin{enumerate} \itemsep -2pt
	\item An example of how the command can be used is: ``rsync -v username@host:$\sim$/path/to/file .''. This commands copies the file at the specified path to the current working directory. The ``-v'' option runs the UNIX command in verbose mode.
	\item Its ``-vr'' option runs the command recursively in verbose mode.
	\item Prof. Aramayo mentioned something about an option that transfers files with automatic compression and decompression. Is this option ``-a'', or something else? Use the ``tar'' command to compress/uncompress files.
	\end{enumerate}
\item script: \vspace{-0.3cm}
	\begin{enumerate} \itemsep -2pt
	\item Use this UNIX command to keep a log of the terminal session. That is, use it to record the terminal session. Some {\tt Terminal} applications allow people to save the terminal session as a text file.
	\item Prof. Aramayo suggested redirecting the standard output stream to a file to keep a record of the commands that are executed and their standard output. However, this requires appending each UNIX command with the redirection symbols.
	\item {\tt unix-command} $>$ zlog \&. This creates a new file, if the file {\tt zlog} does not exist, and redirects the standard output to the logfile ({\tt zlog}). Using the ampersand symbol allows the command to be run in the background while allowing me to continue using the {\tt Terminal} application.
	\item {\tt unix-command} $>>$ zlog \&. This creates a new file, if the file {\tt zlog} does not exist, and redirects the standard output to the logfile ({\tt zlog}). However, if the file {\tt zlog} already exist, the redirected standard output would be appended to the end of the logfile ({\tt zlog}).
	\item Put {\tt \&$>$ slog \&} at the end of each command???
	\end{enumerate}
\item touch
\item tree
\item type: type zrio
\item whatis
\item which: which blastn
\end{enumerate}


Use ``tab'' to autocomplete filenames and directory names. Avoid using spaces in filenames and directories to keep file and directory access simple. \\

Directory access: The ``.'' file is the current working directory, and the ``..'' is the parent directory. A directory can also be called a folder. By using the {\tt cd} command, I can return to my home directory. \\

You cannot undo operations in UNIX. Hence, save and backup files before performing removal operations in UNIX. There is also no ``trash can'' or ``recycle bin''. \\

Microsoft Excel has a maximum limit of 65,000 rows in the spreadsheet. Hence, this limits the amount of information that I can process with Microsoft Excel. To process more data, such as GBs or TBs of data, I need other software applications or develop my own computer program. \\ 

Symbolic links in UNIX are like shortcuts or aliases in Windows. An example of creating a symbolic link is: ``ln -s ../01/test01''. \\

The human genome has been decoded into a file about 7 TB. \\

The colon ``:'' serves as a dummy placeholder to remove the contents of a file; ``: $>$ filename'' \\

Standard output stream, {\tt stdout}, is described along with exit signals of UNIX processes. Standard error output stream will write to the standard error output file. UNIX redirection for standard output and error streams are described. \\

Use {\it tree} to show contents of a directory as a tree. \\

Discussed UNIX path redirection, pipelining of UNIX commands, and separate execution of UNIX commands (using the semicolon ``;'' symbol). \\

Covered special/escape characters to use tabs and newlines to print information. \\

Covered information on how to go to the ``home'' directory. ``$\sim$'' refers to the home directory. \\


Covered absolute paths and relative paths in UNIX. \\

Detailed explanation of the ``ls'' command. It indicates when the file has been created/modified. It also indicates the size of the file in bytes. It also indicates the username (``db0015'') and the group (``student'') that I belong to. Permissions to access files are determined by the group that I belong to. File permissions are indicated for read, write, and execute. They are set for individual users, groups, and everybody with access to the computer network/system. File types are indicated for directories (``d''), regular/normal files (``-''), and symbolic links (``l''). \\

Most files have the file permissions set as 755. \\

Discussed how to create aliases in UNIX. \\

Configure my UNIX environment with the ``.bashrc'' (or ``.bash\_profile'') file.  \\

The {\tt .profile} is used by {\tt shell}, and is equivalent to {\tt .bashrc} for Bash. \\

GUI-based {\it Galaxy} is used for this class. \\


Regarding file transfer, avoid unencrypted file transfer that can be accessed by others. People can listen or snoop on the packet transmission of files, and find out what you are doing. An aside: Email service providers, such as Google, transmit emails between their servers without encryption. \\

There are many applications for downloading files from the Internet. The applications {\tt curl} and {\it wget} are more common for downloading files. \\

The UNIX command {\tt ifconfig} gives you information about computer networking for your computer or computing account (if you are connected to a remote computer). \\





Further references in my research database about UNIX include the following: \cite{Apple2011,Kernighan1984,Kerrisk2010,Mitchell2001,Petersen2008,Raymond2004,Raymond2004a,Rochkind2004,Rosen2007a,Stallings2005,Stevens2013,Storimer2012,VibrantPublishers2010a,VibrantPublishers2011b,VibrantPublishers2011c,VibrantPublishers2011h}.

%%%%%%%%%%%%%%%%%%%%%%%%%%%%%%%%%%%%%%%%%%%%%%
\subsection{SSH Basics}
\label{ssec:SSHBasics}


{\tt SSH} is an application that allows me to connect securely to another computer that is connected to the same computer network, or to the Internet. It uses encryption for network connection, including file transfer between computers in the same network, or between different networks. Its various levels of encryption correspond to various levels of simplicity in the encryption. \\


The actual/real {\tt SSH} application requires paid subscription. However, its open source variant is FREE!!! \\

SSH key generation creates a pair of private and public keys. Keep the private key private to myself (only). Allow others to have the public key, so that a valid authentication of myself can be made. \\










{\tt rsync} is an application for file copying and synchronization between different computer accounts. It does not copy all files in your directory, but copy modifications to existing files and copies only new files. It transfers files in compressed format. That is, it transfer files between different computers by synchronizing them via delta modifications. This is because copying entire directories of huge files take a lot of time. Hence, use {\tt rsync} to carry out file transfer to save time. {\tt rsync} uses the public key of SSH (from SSH key generation) to connect the local machine to the remote machine. For example, I can create the authentication file (SSH public key) and transfer the public key to the remote machine. \\






%%%%%%%%%%%%%%%%%%%%%%%%%%%%%%%%%%%%%%%%%%%%%%
\subsubsection{Shell Scripting Basics}
\label{sssec:ShellScriptingBasics}

To review the basics of UNIX shell script, see my internship report (and associated material) for my internship at the Institute of Microelectronics, Singapore \cite{Ong2004a}.

%%%%%%%%%%%%%%%%%%%%%%%%%%%%%%%%%%%%%%%%%%%%%%
\subsubsection{Related Issues}
\label{sssec:ShellScriptingBasics:RelatedIssues}

Download data to group directory on ``Geiger'', so that I do not corrupt the local machine. \\

Use ``tree'' to find out the directory structure of the specified directory. \\

For class on June 17, 2014, clone the repository from Prof. Aramayo, \url{https://geiger.tamu.edu/gitlab/raramayo/digitalbiology_project_summer2014}. Work on this directory to practise the UNIX sub-lesson for today.






%%%%%%%%%%%%%%%%%%%%%%%%%%%%%%%%%%%%%%%%%%%%%%
\subsection{Version Control (Or Revision Control)}
\label{ssec:Version Control}

Revision control is also known as version control or source control. It is an aspect of software configuration management (SCM). For this class, {\tt Git} \cite{Chacon2009,Swicegood2010,Humble2011,VibrantPublishers2012b,Fox2013} will be our revision control tool. \\

{\tt git status} tells me the status of my {\tt Git} repository. {\tt git diff} tells me the difference between different commits/stages of my repository. Watch videos about {\tt Git} to learn more about {\tt Git}, via hyperlinks provided on the class Wiki. Also, read ``{\tt Git} in the Trenches.'' \\

While adding files to my {\tt Git} repository, use the {\tt Markdown} language to provide some structure to the presentation of information for my project repository. Save files in the {\tt Markdown} language as {\tt filename.md}. {\tt Markdown} is a document markup language, just like \LaTeX, HTML, and XML. \\

%%%%%%%%%%%%%%%%%%%%%%%%%%%%%%%%%%%%%%%%%%%%%%
\section{Other Computing Issues}
\label{sec:OtherComputingIssues}


%	genomics01.bio.tamu.edu
%	usr:	dbiology
%	passwd (integralmente)


Launch system monitor to track how much CPU time are processes taking. 





















