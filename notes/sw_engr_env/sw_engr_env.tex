%	This LaTeX file is written by Zhiyang Ong to record notes for his digital biology class regarding the basics of software engineering and the UNIX environment.

%	The MIT License (MIT)

%	Copyright (c) <2014> <Zhiyang Ong>

%	Permission is hereby granted, free of charge, to any person obtaining a copy of this software and associated documentation files (the "Software"), to deal in the Software without restriction, including without limitation the rights to use, copy, modify, merge, publish, distribute, sublicense, and/or sell copies of the Software, and to permit persons to whom the Software is furnished to do so, subject to the following conditions:

%	The above copyright notice and this permission notice shall be included in all copies or substantial portions of the Software.

%	THE SOFTWARE IS PROVIDED "AS IS", WITHOUT WARRANTY OF ANY KIND, EXPRESS OR IMPLIED, INCLUDING BUT NOT LIMITED TO THE WARRANTIES OF MERCHANTABILITY, FITNESS FOR A PARTICULAR PURPOSE AND NONINFRINGEMENT. IN NO EVENT SHALL THE AUTHORS OR COPYRIGHT HOLDERS BE LIABLE FOR ANY CLAIM, DAMAGES OR OTHER LIABILITY, WHETHER IN AN ACTION OF CONTRACT, TORT OR OTHERWISE, ARISING FROM, OUT OF OR IN CONNECTION WITH THE SOFTWARE OR THE USE OR OTHER DEALINGS IN THE SOFTWARE.

%	Email address: echo "cukj -wb- 23wU4X5M589 TROJANS cqkH wiuz2y 0f Mw Stanford" | awk '{ sub("23wU4X5M589","F.d_c_b. ") sub("Stanford","d0mA1n"); print $5, $2, $8; for (i=1; i<=1; i++) print "6\b"; print $9, $7, $6 }' | sed y/kqcbuHwM62z/gnotrzadqmC/ | tr 'q' ' ' | tr -d [:cntrl:] | tr -d 'ir' | tr y "\n"

%%%%%%%%%%%%%%%%%%%%%%%%%%%%%%%%%%%%%%%%%%%%%%





%%%%%%%%%%%%%%%%%%%%%%%%%%%%%%%%%%%%%%%%%%%%%%
\chapter{Software Engineering Basics}
\label{chp:SWEngrBasics}


%%%%%%%%%%%%%%%%%%%%%%%%%%%%%%%%%%%%%%%%%%%%%%
\section{UNIX Basics}
\label{sec:UNIXBasics}

Terminal 

UNIX philosophy

Operate in the UNIX environment via text files. Everything is a file in UNIX. Some files can be read visually, while others cannot.

Anatomy of UNIX commands: {\tt command\_name [options] [arguments]}. Double dashes for options of UNIX commands cannot be combined. However, for options for single dash lines, they can be combined.

The ``man'' page is the UNIX manual. To find documentation of a UNIX command, use the 

{\tt SSH} is an application that allows me to connect securely to another computer that is connected to the same computer network, or to the Internet.

{\tt rsync} is an application for file copying and synchronization between different computer accounts. It does not copy all files in your directory, but copy modifications to existing files and copies only new files. It transfers files in compressed format.



UNIX commands to learn: \vspace{-0.3cm}
\begin{enumerate}	\itemsep -4pt
	\item alias: ``alias ll''
	\item apropos: ``apropos copy'' would search the UNIX ``man '' pages for the keyword ``copy''.
	\item cat
	\item cd
	\item chmod: Change mode
	\item clear
	\item cp: cp --version
	\item dir -l
	\item date
	\item du: ``du -hd 0 .'' list the size of the directory in KB, and ``du -hd 1 .'' list the size of the directory and its files. ``df -h'' indicates the size of the directory and its contents. 
	\item echo: ``echo -e'' refers to {\tt echo} enhanced, which redirects the output in the UNIX pipeline to a file. {\it echo -e "`date`" > tata1}. ``echo \$PATH''
	\item file
	\item history
	\item info cp
	\item less
	\item ls [-al]
	\item more
	\item mkdir
	\item mv
	\item pwd
	\item rm
	\item rmdir
	\item rsync. An example of how the command can be used is: ``rsync -v username@host:~/path/to/file .''. The ``-v'' option runs the UNIX command in verbose mode.
	\item touch
	\item tree
	\item type: type zrio
	\item whatis
	\item which: which blastn
\end{enumerate}


Use ``tab'' to autocomplete filenames and directory names. Avoid using spaces in filenames and directories to keep file and directory access simple.

Directory access: The ``.'' file is the current working directory, and the ``..'' is the parent directory. A directory can also be called a folder. By using the {\tt cd}

You cannot undo operations in UNIX. Hence, save and backup files before performing removal operations in UNIX. There is also no ``trash can'' or ``recycle bin''.

Microsoft Excel has a maximum limit of 65,000 rows in the spreadsheet.

Symbolic links in UNIX are like shortcuts or aliases in Windows. An example of creating a symbolic link is: ``ln -s ../01/test01''.

The human genome has been decoded into a file about 7 TB.

The colon ``:'' serves as a dummy placeholder to remove the contents of a file; ``: > filename''

Standard output stream, {\tt stdout}, is described along with exit signals of UNIX processes. Standard error output stream will write to the standard error output file. UNIX redirection for standard output and error streams are described.

Use {\it tree} to show contents of a directory as a tree.

Discussed UNIX path redirection, pipelining of UNIX commands, and separate execution of UNIX commands (using the semicolon ``;'' symbol).

Covered special/escape characters to use tabs and newlines to print information.

Covered information on how to go to the ``home'' directory. ``\sim'' refers to the home directory.


Covered absolute paths and relative paths in UNIX.

Detailed explanation of the ``ls'' command. It indicates when the file has been created/modified. It also indicates the size of the file in bytes. It also indicates the username (``db0015'') and the group (``student'') that I belong to. Permissions to access files are determined by the group that I belong to. File permissions are indicated for read, write, and execute. They are set for individual users, groups, and everybody with access to the computer network/system. File types are indicated for directories (``d''), regular/normal files (``-''), and symbolic links (``l'').

Most files have the file permissions set as 755.

Discussed how to create aliases in UNIX.

Configure my UNIX environment with the ``.bashrc'' (or ``.bash_profile'') file.

GUI-based {\it Galaxy} is used for this class.








%%%%%%%%%%%%%%%%%%%%%%%%%%%%%%%%%%%%%%%%%%%%%%
\subsection{SSH Basics}
\label{ssec:SSHBasics}

File 






There are many applications for downloading files from the Internet. The applications {\tt curl} and {\it wget} are more common for downloading files.

The UNIX command {\tt ifconfig} gives you information about computer networking for your computer or computing account (if you are connected to a remote computer).










































