%	This LaTeX file is written by Zhiyang Ong to record notes for his digital biology class regarding the introductory material.

%	The MIT License (MIT)

%	Copyright (c) <2014> <Zhiyang Ong>

%	Permission is hereby granted, free of charge, to any person obtaining a copy of this software and associated documentation files (the "Software"), to deal in the Software without restriction, including without limitation the rights to use, copy, modify, merge, publish, distribute, sublicense, and/or sell copies of the Software, and to permit persons to whom the Software is furnished to do so, subject to the following conditions:

%	The above copyright notice and this permission notice shall be included in all copies or substantial portions of the Software.

%	THE SOFTWARE IS PROVIDED "AS IS", WITHOUT WARRANTY OF ANY KIND, EXPRESS OR IMPLIED, INCLUDING BUT NOT LIMITED TO THE WARRANTIES OF MERCHANTABILITY, FITNESS FOR A PARTICULAR PURPOSE AND NONINFRINGEMENT. IN NO EVENT SHALL THE AUTHORS OR COPYRIGHT HOLDERS BE LIABLE FOR ANY CLAIM, DAMAGES OR OTHER LIABILITY, WHETHER IN AN ACTION OF CONTRACT, TORT OR OTHERWISE, ARISING FROM, OUT OF OR IN CONNECTION WITH THE SOFTWARE OR THE USE OR OTHER DEALINGS IN THE SOFTWARE.

%	Email address: echo "cukj -wb- 23wU4X5M589 TROJANS cqkH wiuz2y 0f Mw Stanford" | awk '{ sub("23wU4X5M589","F.d_c_b. ") sub("Stanford","d0mA1n"); print $5, $2, $8; for (i=1; i<=1; i++) print "6\b"; print $9, $7, $6 }' | sed y/kqcbuHwM62z/gnotrzadqmC/ | tr 'q' ' ' | tr -d [:cntrl:] | tr -d 'ir' | tr y "\n"

%%%%%%%%%%%%%%%%%%%%%%%%%%%%%%%%%%%%%%%%%%%%%%


%%%%%%%%%%%%%%%%%%%%%%%%%%%%%%%%%%%%%%%%%%%%%%
\chapter{Introuctory Material}
\label{chp:SWEngrBasics}

Prof. Rodolfo Aramayo is my class instructor.

This is a UNIX-based class. \\

My username is db00XX. See the comments of this statement for my username. \\ % db0015

%	la00rpe		laoo,rpe		la00,rpe
%	rogofo

Ricardo is my (lab) teaching assistant (TA). \\ 

Genome assembly is still an unsolved problem. \\

{\it GitLab} (from GitHub, Inc.) will be used for the first time in this class. Three concepts will be covered in the introduction: Wiki for the class, which contains the standard class information; the code repository that the Wiki uses; and the {\it Git} version control system. \\

The outline of the class (i.e., syllabus and class schedule) will be modified as the semester progresses. \\

On Thursday, June 5, 2014, we will cover genome analysis, gene models, and gene files. Next week, we will cover next generation DNA sequencing. We will also look at library construction methodology and techniques, and associated challenges. Next Thursday, we will also look at ``small reads.'' We will write small scripts to process small data sets, and organize the pipeline (or design the algorithm) for the program/script. That is, design the control and data- flow graph of the algorithm. Furthermore, we will look at genome mapping, genome assembly, data display (i.e., data visualization), and transcriptome mapping and transcriptome assembly. Subsequently, we will be given data sets from the professor to carry out (machine) learning for pattern classification, and explore read archives (with unknown outputs). \\




